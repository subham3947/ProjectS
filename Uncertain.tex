%%%%%%%%%%%%%%%%%%%%%%%%%%%%%%%%%%%%%%%%%%%%%%%%%%%%%%%%%%%%%%%%%%%%%%%%%%%%%%%%
%2345678901234567890123456789012345678901234567890123456789012345678901234567890
%        1         2         3         4         5         6         7         8

%documentclass[letterpaper, 10 pt, conference]{ieeeconf}  % Comment this line out
                                                          % if you need a4paper
\documentclass[a4paper, 10pt, conference]{ieeeconf}      % Use this line for a4
                                                          % paper

% See the \addtolength command later in the file to balance the column lengths
% on the last page of the document



% The following packages can be found on http:\\www.ctan.org
%\usepackage{graphics} % for pdf, bitmapped graphics files
%\usepackage{epsfig} % for postscript graphics files
%\usepackage{mathptmx} % assumes new font selection scheme installed
%\usepackage{times} % assumes new font selection scheme installed
%\usepackage{amsmath} % assumes amsmath package installed
%\usepackage{amssymb}  % assumes amsmath package installed

\title{\LARGE \bf
Preparation of Papers for IEEE Sponsored Conferences \& Symposia*
}

%\author{ \parbox{3 in}{\centering Huibert Kwakernaak*
%         \thanks{*Use the $\backslash$thanks command to put information here}\\
%         Faculty of Electrical Engineering, Mathematics and Computer Science\\
%         University of Twente\\
%         7500 AE Enschede, The Netherlands\\
%         {\tt\small h.kwakernaak@autsubmit.com}}
%         \hspace*{ 0.5 in}
%         \parbox{3 in}{ \centering Pradeep Misra**
%         \thanks{**The footnote marks may be inserted manually}\\
%        Department of Electrical Engineering \\
%         Wright State University\\
%         Dayton, OH 45435, USA\\
%         {\tt\small pmisra@cs.wright.edu}}
%}

\author{Huibert Kwakernaak$^{1}$ and Pradeep Misra$^{2}$% <-this % stops a space
}


\begin{document}



\maketitle
\thispagestyle{empty}
\pagestyle{empty}


%%%%%%%%%%%%%%%%%%%%%%%%%%%%%%%%%%%%%%%%%%%%%%%%%%%%%%%%%%%%%%%%%%%%%%%%%%%%%%%%
\begin{abstract}

This electronic document is a ÒliveÓ template. The various components of your paper [title, text, heads, etc.] are already defined on the style sheet, as illustrated by the portions given in this document.

\end{abstract}


%%%%%%%%%%%%%%%%%%%%%%%%%%%%%%%%%%%%%%%%%%%%%%%%%%%%%%%%%%%%%%%%%%%%%%%%%%%%%%%%
\section{INTRODUCTION}

This template, modified in MS Word 2003 and saved as ÒWord 97-2003 \& 6.0/95 Ð RTFÓ for the PC, provides authors with most of the formatting specifications needed for preparing electronic versions of their papers. All standard paper components have been specified for three reasons: (1) ease of use when formatting individual papers, (2) automatic compliance to electronic requirements that facilitate the concurrent or later production of electronic products, and (3) conformity of style throughout a conference proceedings. Margins, column widths, line spacing, and type styles are built-in; examples of the type styles are provided throughout this document and are identified in italic type, within parentheses, following the example. Some components, such as multi-leveled equations, graphics, and tables are not prescribed, although the various table text styles are provided. The formatter will need to create these components, incorporating the applicable criteria that follow.

\subsection{Selecting a Template (Heading 2)}

First, confirm that you have the correct template for your paper size. This template has been tailored for output on the US-letter paper size. Please do not use it for A4 paper since the margin requirements for A4 papers may be different from Letter paper size.

\subsection{Maintaining the Integrity of the Specifications}

The template is used to format your paper and style the text. All margins, column widths, line spaces, and text fonts are prescribed; please do not alter them. You may note peculiarities. For example, the head margin in this template measures proportionately more than is customary. This measurement and others are deliberate, using specifications that anticipate your paper as one part of the entire proceedings, and not as an independent document. Please do not revise any of the current designations

\section{RELATED WORK}

Before you begin to format your paper, first write and save the content as a separate text file. Keep your text and graphic files separate until after the text has been formatted and styled. Do not use hard tabs, and limit use of hard returns to only one return at the end of a paragraph. Do not add any kind of pagination anywhere in the paper. Do not number text heads-the template will do that for you.

Finally, complete content and organizational editing before formatting. Please take note of the following items when proofreading spelling and grammar:

\subsection{Abbreviations and Acronyms} Define abbreviations and acronyms the first time they are used in the text, even after they have been defined in the abstract. Abbreviations such as IEEE, SI, MKS, CGS, sc, dc, and rms do not have to be defined. Do not use abbreviations in the title or heads unless they are unavoidable.

\subsection{Equations}
The equations are an exception to the prescribed specifications of this template. You will need to determine whether or not your equation should be typed using either the Times New Roman or the Symbol font (please no other font). To create multileveled equations, it may be necessary to treat the equation as a graphic and insert it into the text after your paper is styled. Number equations consecutively. Equation numbers, within parentheses, are to position flush right, as in (1), using a right tab stop. To make your equations more compact, you may use the solidus ( / ), the exp function, or appropriate exponents. Italicize Roman symbols for quantities and variables, but not Greek symbols. Use a long dash rather than a hyphen for a minus sign. Punctuate equations with commas or periods when they are part of a sentence, as in

$$
\alpha + \beta = \chi \eqno{(1)}
$$

Note that the equation is centered using a center tab stop. Be sure that the symbols in your equation have been defined before or immediately following the equation. Use Ò(1)Ó, not ÒEq. (1)Ó or Òequation (1)Ó, except at the beginning of a sentence: ÒEquation (1) is . . .Ó

\subsection{Some Common Mistakes}
\begin{itemize}


\item The word ÒdataÓ is plural, not singular.
\item The subscript for the permeability of vacuum ?0, and other common scientific constants, is zero with subscript formatting, not a lowercase letter ÒoÓ.
\item In American English, commas, semi-/colons, periods, question and exclamation marks are located within quotation marks only when a complete thought or name is cited, such as a title or full quotation. When quotation marks are used, instead of a bold or italic typeface, to highlight a word or phrase, punctuation should appear outside of the quotation marks. A parenthetical phrase or statement at the end of a sentence is punctuated outside of the closing parenthesis (like this). (A parenthetical sentence is punctuated within the parentheses.)
\item A graph within a graph is an ÒinsetÓ, not an ÒinsertÓ. The word alternatively is preferred to the word ÒalternatelyÓ (unless you really mean something that alternates).
\item Do not use the word ÒessentiallyÓ to mean ÒapproximatelyÓ or ÒeffectivelyÓ.
\item In your paper title, if the words Òthat usesÓ can accurately replace the word ÒusingÓ, capitalize the ÒuÓ; if not, keep using lower-cased.
\item Be aware of the different meanings of the homophones ÒaffectÓ and ÒeffectÓ, ÒcomplementÓ and ÒcomplimentÓ, ÒdiscreetÓ and ÒdiscreteÓ, ÒprincipalÓ and ÒprincipleÓ.
\item Do not confuse ÒimplyÓ and ÒinferÓ.
\item The prefix ÒnonÓ is not a word; it should be joined to the word it modifies, usually without a hyphen.
\item There is no period after the ÒetÓ in the Latin abbreviation Òet al.Ó.
\item The abbreviation Òi.e.Ó means Òthat isÓ, and the abbreviation Òe.g.Ó means Òfor exampleÓ.

\end{itemize}


\section{PROPOSED METHODOLOGY}
Our approach consists of two tasks: first is the creation of a domain-specific polarity lexicon with a polarity weight assigned to each word as positive,negative or neutral using a news corpora. Second is proposing a sentiment calculator to evaluate the created lexicon.
\subsection{Domain-Specific Lexicon Generation}
In our study we are creating the domain specific lexicon with the help of SentiWordNet 3.0 \cite{c1}.SentiWordNet 3.0 is the most famous and popular lexicon for sentiment analysis. It contains more than 1,17,600 words  there meanings, part of speech  and the degree of positivity and negativity of the word, ranging from 0 to 1.

We create a bag of words from a corpus of 417 documents all belonging to a particular(financial) domain tagged with their part of speech.Then we add the words along wih their part-of-speech and their degree of positivity or negativity in our lexicon.We have also taken care of the fact that a word can be expressed in more than one sense or part of speech.The sense which is only appropriate to our domain has been selected.For example, \textbf{``critical''} being an adjective is represented in total 7 senses in SentiWordNet 3.0 comprising of positive,negative and neutral polarity. The first sense of the word \textbf{``critical''} represents ``marked by a tendency to find and call attention to errors and flaws'' having 0.5 negative polarity. The second sense shows neutral character of the word by meaning ``at or of a point at which a property or phenomenon suffers an abrupt change especially having enough mass to sustain a chain reaction'' while the positive sense is depicted by ``forming or having the nature of a turning point or crisis''. So we have selected the 6th sense of the word which suits our domain with 0.125 negative score that means "being in or verging on a state of crisis or emergency".

\subsubsection{Non-neutral Words and Domain Specific Scoring}
After the addition of words with their respective polarity,part of speech and sense we end up with 4890 words in our lexicon among which most of them are neutral words. But the words that are termed as neutral by SentiWordNet 3.0 may differ in polarity in a specific domain. For example, \textbf{``growth''} in SentiWordNet 3.0 is neutral in all senses but in our domain ``growth'' mostly appears with ``GDP growth'',``economic growth'',``revenue growth accelerated'' or ``credit growth''. Likewise the word \textbf{``NPA''} is neutral in SentiWordNet 3.0 being an abbreviation for ``new people's army'' whereas in a financial domain it stands for Non-Performing Asset. Similarly, words like inflation, recession, dwindling, lagging, losses, obstacle, shutdown, protest, etc. all are termed as neutral by SentiWordNet 3.0. There are 85 such words which are not neutral in our domain.

We adopt a simple and popular information retrieval technique to assign score to the words that are not neutral in our domain called \textbf{TF-IDF} (Term Frequency-Inverse
Document Frequency)\cite{c7}. Term frequency is defined as the total number of times a given word appears
in a document divided by the total number of words in the document. It may also be defined as the
frequency of the occurrence of certain terms in a given document. Inverse Document Frequency is defined as the
document count that lies in the corpus in which a given term coexists.
It can be computed by finding the logarithm of the total number of documents
present in a given corpus divided by the number of documents in which a particular
token exists.

Term Frequency\textbf{(TF)} is calculated as follows:

	tf(term, document)=(word, document)/ length(document)

	
Inverse Document Frequency \textbf{(IDF)} is calculated as follows:

	idf(term,corpus)= log(length(corpus)/ 	count(document containing term, corpus)


The \textbf{TF-IDF} score can be obtained by multiplying both scores. 

	\textbf{TF-IDF}(term, document, corpus) = TF(term ,document)
	 * IDF(term, corpus)
	 
After classifying those neutral words as positive or negative we add them to our lexicon with their respective TF-IDF scores.

\subsection{Sentiment Intensity Calculator}
Our main objective is to measure the accuracy of the proposed domain-specific lexicon. We propose an algorithm to analyze the sentiment of a sentence by bigram approach. The selection of the bigrams from a sentence for analysing the sentiment depends upon the part of speech which can be polar or can have polarity(mostly adjectives, nouns, verbs and adverbs). For POS tagging we use Stanford NLP Tagger \cite{c8}.

\section{EXPERIMENT}

\subsection{Dataset}

The datasets used for our study were of two different types on a particular domain. The first one being the newspaper articles from popular Indian newspapers relating to GST(Goods And Services Tax) . The second dataset comprised of the tweets that depict the public opinion on GST for the month of July, 2017.

\subsubsection{News Corpora}

The data was acquired from four popular Indian newspapers namely, The Economic Times \cite{c1}, The Hindu \cite{c2}, The Tribune \cite{c3} and The Statesman \cite{c4} from  1st July, 2017 (GST launch) till 31st August, 2017. The articles were scraped by using a popular Python library called Newspaper3k \cite{c5}. The total no. of news articles acquired on GST for two months(1st July, 2017 to 31st August, 2017) were 417.

\subsubsection{Twitter Dataset}
This dataset was created from Twitter, a popular social networking service where users post their messages and opinion called tweets.The tweets collected were based on GST which can depict the public sentiment after the launch of GST in India. The timeline of the collected tweets was from 1st July, 2017(GST launch) till 31st July, 2017.


\subsection{Evaluation}
The domain-specific lexicon created from the news corpus was evaluated on the twitter dataset by using the sentiment measurer introduced in Section 0.0. The sentiment of each tweet was measured using both the domain-specific and SentiWordNet 3.0 lexicon to compare the difference between the intensities of the sentiment values by the two lexicon.

\section{CONCLUSIONS}

A conclusion section is not required. Although a conclusion may review the main points of the paper, do not replicate the abstract as the conclusion. A conclusion might elaborate on the importance of the work or suggest applications and extensions. 

\addtolength{\textheight}{-12cm}   % This command serves to balance the column lengths
                                  % on the last page of the document manually. It shortens
                                  % the textheight of the last page by a suitable amount.
                                  % This command does not take effect until the next page
                                  % so it should come on the page before the last. Make
                                  % sure that you do not shorten the textheight too much.

%%%%%%%%%%%%%%%%%%%%%%%%%%%%%%%%%%%%%%%%%%%%%%%%%%%%%%%%%%%%%%%%%%%%%%%%%%%%%%%%



%%%%%%%%%%%%%%%%%%%%%%%%%%%%%%%%%%%%%%%%%%%%%%%%%%%%%%%%%%%%%%%%%%%%%%%%%%%%%%%%



%%%%%%%%%%%%%%%%%%%%%%%%%%%%%%%%%%%%%%%%%%%%%%%%%%%%%%%%%%%%%%%%%%%%%%%%%%%%%%%%
\section*{APPENDIX}

Appendixes should appear before the acknowledgment.

\section*{ACKNOWLEDGMENT}

The preferred spelling of the word ÒacknowledgmentÓ in America is without an ÒeÓ after the ÒgÓ. Avoid the stilted expression, ÒOne of us (R. B. G.) thanks . . .Ó  Instead, try ÒR. B. G. thanksÓ. Put sponsor acknowledgments in the unnumbered footnote on the first page.

Use this sample document as your LaTeX source file to create your document. Save this file as {\bf root.tex}. You have to make sure to use the cls file that came with this distribution. If you use a different style file, you cannot expect to get required margins. Note also that when you are creating your out PDF file, the source file is only part of the equation. {\it Your \TeX\ $\rightarrow$ PDF filter determines the output file size. Even if you make all the specifications to output a letter file in the source - if you filter is set to produce A4, you will only get A4 output. }

It is impossible to account for all possible situation, one would encounter using \TeX. If you are using multiple \TeX\ files you must make sure that the ``MAIN`` source file is called root.tex - this is particularly important if your conference is using PaperPlaza's built in \TeX\ to PDF conversion tool.

\subsection{Headings, etc}

Text heads organize the topics on a relational, hierarchical basis. For example, the paper title is the primary text head because all subsequent material relates and elaborates on this one topic. If there are two or more sub-topics, the next level head (uppercase Roman numerals) should be used and, conversely, if there are not at least two sub-topics, then no subheads should be introduced. Styles named ÒHeading 1Ó, ÒHeading 2Ó, ÒHeading 3Ó, and ÒHeading 4Ó are prescribed.

\subsection{Figures and Tables}

Positioning Figures and Tables: Place figures and tables at the top and bottom of columns. Avoid placing them in the middle of columns. Large figures and tables may span across both columns. Figure captions should be below the figures; table heads should appear above the tables. Insert figures and tables after they are cited in the text. Use the abbreviation ÒFig. 1Ó, even at the beginning of a sentence.

\begin{table}[h]
	\caption{An Example of a Table}
	\label{table_example}
	\begin{center}
		\begin{tabular}{|c||c|}
			\hline
			One & Two\\
			\hline
			Three & Four\\
			\hline
		\end{tabular}
	\end{center}
\end{table}


\begin{figure}[thpb]
	\centering
	\framebox{\parbox{3in}{We suggest that you use a text box to insert a graphic (which is ideally a 300 dpi TIFF or EPS file, with all fonts embedded) because, in an document, this method is somewhat more stable than directly inserting a picture.
	}}
	%\includegraphics[scale=1.0]{figurefile}
	\caption{Inductance of oscillation winding on amorphous
		magnetic core versus DC bias magnetic field}
	\label{figurelabel}
\end{figure}


Figure Labels: Use 8 point Times New Roman for Figure labels. Use words rather than symbols or abbreviations when writing Figure axis labels to avoid confusing the reader. As an example, write the quantity ÒMagnetizationÓ, or ÒMagnetization, MÓ, not just ÒMÓ. If including units in the label, present them within parentheses. Do not label axes only with units. In the example, write ÒMagnetization (A/m)Ó or ÒMagnetization {A[m(1)]}Ó, not just ÒA/mÓ. Do not label axes with a ratio of quantities and units. For example, write ÒTemperature (K)Ó, not ÒTemperature/K.Ó



%%%%%%%%%%%%%%%%%%%%%%%%%%%%%%%%%%%%%%%%%%%%%%%%%%%%%%%%%%%%%%%%%%%%%%%%%%%%%%%%

References are important to the reader; therefore, each citation must be complete and correct. If at all possible, references should be commonly available publications.



\begin{thebibliography}{99}
\bibitem{c1}
Baccianella, Stefano and Esuli, Andrea and Sebastiani, Fabrizio. (2010). SentiWordNet 3.0: An Enhanced Lexical Resource for Sentiment Analysis and Opinion Mining.. Proceedings of LREC. 10.
\bibitem{c2} 
https://economictimes.indiatimes.com/
\bibitem{c3} 
http://www.thehindu.com/
\bibitem{c4}
http://www.tribuneindia.com/
\bibitem{c5} 
https://www.thestatesman.com/
\bibitem{c6} http://newspaper.readthedocs.io/en/latest/
\bibitem{c7}Deepti Chopra,Nisheeth Joshi,Iti Mathur, Mastering Natural Language Processing with Python, 2016.
\bibitem{c8} Kristina Toutanova, Dan Klein, Christopher Manning, and Yoram Singer. 2003. Feature-Rich Part-of-Speech Tagging with a Cyclic Dependency Network. In Proceedings of HLT-NAACL 2003, pp. 252-259.
\bibitem{c9} M. Young, The Techincal Writers Handbook.  Mill Valley, CA: University Science, 1989.
\bibitem{c10} J. U. Duncombe, ÒInfrared navigationÑPart I: An assessment of feasibility (Periodical style),Ó IEEE Trans. Electron Devices, vol. ED-11, pp. 34Ð39, Jan. 1959.
\bibitem{c11} S. Chen, B. Mulgrew, and P. M. Grant, ÒA clustering technique for digital communications channel equalization using radial basis function networks,Ó IEEE Trans. Neural Networks, vol. 4, pp. 570Ð578, July 1993.
\bibitem{c12} R. W. Lucky, ÒAutomatic equalization for digital communication,Ó Bell Syst. Tech. J., vol. 44, no. 4, pp. 547Ð588, Apr. 1965.
\bibitem{c13} S. P. Bingulac, ÒOn the compatibility of adaptive controllers (Published Conference Proceedings style),Ó in Proc. 4th Annu. Allerton Conf. Circuits and Systems Theory, New York, 1994, pp. 8Ð16.
\bibitem{c14} G. R. Faulhaber, ÒDesign of service systems with priority reservation,Ó in Conf. Rec. 1995 IEEE Int. Conf. Communications, pp. 3Ð8.
\bibitem{c15} W. D. Doyle, ÒMagnetization reversal in films with biaxial anisotropy,Ó in 1987 Proc. INTERMAG Conf., pp. 2.2-1Ð2.2-6.
\bibitem{c16} G. W. Juette and L. E. Zeffanella, ÒRadio noise currents n short sections on bundle conductors (Presented Conference Paper style),Ó presented at the IEEE Summer power Meeting, Dallas, TX, June 22Ð27, 1990, Paper 90 SM 690-0 PWRS.
\bibitem{c17} J. G. Kreifeldt, ÒAn analysis of surface-detected EMG as an amplitude-modulated noise,Ó presented at the 1989 Int. Conf. Medicine and Biological Engineering, Chicago, IL.
\bibitem{c18} J. Williams, ÒNarrow-band analyzer (Thesis or Dissertation style),Ó Ph.D. dissertation, Dept. Elect. Eng., Harvard Univ., Cambridge, MA, 1993. 
\bibitem{c19} N. Kawasaki, ÒParametric study of thermal and chemical nonequilibrium nozzle flow,Ó M.S. thesis, Dept. Electron. Eng., Osaka Univ., Osaka, Japan, 1993.
\bibitem{c20} J. P. Wilkinson, ÒNonlinear resonant circuit devices (Patent style),Ó U.S. Patent 3 624 12, July 16, 1990. 






\end{thebibliography}




\end{document}
